% Options for packages loaded elsewhere
\PassOptionsToPackage{unicode}{hyperref}
\PassOptionsToPackage{hyphens}{url}
%
\documentclass[
]{article}
\usepackage{amsmath,amssymb}
\usepackage{lmodern}
\usepackage{iftex}
\ifPDFTeX
  \usepackage[T1]{fontenc}
  \usepackage[utf8]{inputenc}
  \usepackage{textcomp} % provide euro and other symbols
\else % if luatex or xetex
  \usepackage{unicode-math}
  \defaultfontfeatures{Scale=MatchLowercase}
  \defaultfontfeatures[\rmfamily]{Ligatures=TeX,Scale=1}
\fi
% Use upquote if available, for straight quotes in verbatim environments
\IfFileExists{upquote.sty}{\usepackage{upquote}}{}
\IfFileExists{microtype.sty}{% use microtype if available
  \usepackage[]{microtype}
  \UseMicrotypeSet[protrusion]{basicmath} % disable protrusion for tt fonts
}{}
\makeatletter
\@ifundefined{KOMAClassName}{% if non-KOMA class
  \IfFileExists{parskip.sty}{%
    \usepackage{parskip}
  }{% else
    \setlength{\parindent}{0pt}
    \setlength{\parskip}{6pt plus 2pt minus 1pt}}
}{% if KOMA class
  \KOMAoptions{parskip=half}}
\makeatother
\usepackage{xcolor}
\IfFileExists{xurl.sty}{\usepackage{xurl}}{} % add URL line breaks if available
\IfFileExists{bookmark.sty}{\usepackage{bookmark}}{\usepackage{hyperref}}
\hypersetup{
  pdftitle={UE20CS312 - Data Analytics - Worksheet 2a - Simple Linear Regression},
  pdfauthor={GAURAV MAHAJAN},
  hidelinks,
  pdfcreator={LaTeX via pandoc}}
\urlstyle{same} % disable monospaced font for URLs
\usepackage[margin=1in]{geometry}
\usepackage{color}
\usepackage{fancyvrb}
\newcommand{\VerbBar}{|}
\newcommand{\VERB}{\Verb[commandchars=\\\{\}]}
\DefineVerbatimEnvironment{Highlighting}{Verbatim}{commandchars=\\\{\}}
% Add ',fontsize=\small' for more characters per line
\usepackage{framed}
\definecolor{shadecolor}{RGB}{248,248,248}
\newenvironment{Shaded}{\begin{snugshade}}{\end{snugshade}}
\newcommand{\AlertTok}[1]{\textcolor[rgb]{0.94,0.16,0.16}{#1}}
\newcommand{\AnnotationTok}[1]{\textcolor[rgb]{0.56,0.35,0.01}{\textbf{\textit{#1}}}}
\newcommand{\AttributeTok}[1]{\textcolor[rgb]{0.77,0.63,0.00}{#1}}
\newcommand{\BaseNTok}[1]{\textcolor[rgb]{0.00,0.00,0.81}{#1}}
\newcommand{\BuiltInTok}[1]{#1}
\newcommand{\CharTok}[1]{\textcolor[rgb]{0.31,0.60,0.02}{#1}}
\newcommand{\CommentTok}[1]{\textcolor[rgb]{0.56,0.35,0.01}{\textit{#1}}}
\newcommand{\CommentVarTok}[1]{\textcolor[rgb]{0.56,0.35,0.01}{\textbf{\textit{#1}}}}
\newcommand{\ConstantTok}[1]{\textcolor[rgb]{0.00,0.00,0.00}{#1}}
\newcommand{\ControlFlowTok}[1]{\textcolor[rgb]{0.13,0.29,0.53}{\textbf{#1}}}
\newcommand{\DataTypeTok}[1]{\textcolor[rgb]{0.13,0.29,0.53}{#1}}
\newcommand{\DecValTok}[1]{\textcolor[rgb]{0.00,0.00,0.81}{#1}}
\newcommand{\DocumentationTok}[1]{\textcolor[rgb]{0.56,0.35,0.01}{\textbf{\textit{#1}}}}
\newcommand{\ErrorTok}[1]{\textcolor[rgb]{0.64,0.00,0.00}{\textbf{#1}}}
\newcommand{\ExtensionTok}[1]{#1}
\newcommand{\FloatTok}[1]{\textcolor[rgb]{0.00,0.00,0.81}{#1}}
\newcommand{\FunctionTok}[1]{\textcolor[rgb]{0.00,0.00,0.00}{#1}}
\newcommand{\ImportTok}[1]{#1}
\newcommand{\InformationTok}[1]{\textcolor[rgb]{0.56,0.35,0.01}{\textbf{\textit{#1}}}}
\newcommand{\KeywordTok}[1]{\textcolor[rgb]{0.13,0.29,0.53}{\textbf{#1}}}
\newcommand{\NormalTok}[1]{#1}
\newcommand{\OperatorTok}[1]{\textcolor[rgb]{0.81,0.36,0.00}{\textbf{#1}}}
\newcommand{\OtherTok}[1]{\textcolor[rgb]{0.56,0.35,0.01}{#1}}
\newcommand{\PreprocessorTok}[1]{\textcolor[rgb]{0.56,0.35,0.01}{\textit{#1}}}
\newcommand{\RegionMarkerTok}[1]{#1}
\newcommand{\SpecialCharTok}[1]{\textcolor[rgb]{0.00,0.00,0.00}{#1}}
\newcommand{\SpecialStringTok}[1]{\textcolor[rgb]{0.31,0.60,0.02}{#1}}
\newcommand{\StringTok}[1]{\textcolor[rgb]{0.31,0.60,0.02}{#1}}
\newcommand{\VariableTok}[1]{\textcolor[rgb]{0.00,0.00,0.00}{#1}}
\newcommand{\VerbatimStringTok}[1]{\textcolor[rgb]{0.31,0.60,0.02}{#1}}
\newcommand{\WarningTok}[1]{\textcolor[rgb]{0.56,0.35,0.01}{\textbf{\textit{#1}}}}
\usepackage{graphicx}
\makeatletter
\def\maxwidth{\ifdim\Gin@nat@width>\linewidth\linewidth\else\Gin@nat@width\fi}
\def\maxheight{\ifdim\Gin@nat@height>\textheight\textheight\else\Gin@nat@height\fi}
\makeatother
% Scale images if necessary, so that they will not overflow the page
% margins by default, and it is still possible to overwrite the defaults
% using explicit options in \includegraphics[width, height, ...]{}
\setkeys{Gin}{width=\maxwidth,height=\maxheight,keepaspectratio}
% Set default figure placement to htbp
\makeatletter
\def\fps@figure{htbp}
\makeatother
\setlength{\emergencystretch}{3em} % prevent overfull lines
\providecommand{\tightlist}{%
  \setlength{\itemsep}{0pt}\setlength{\parskip}{0pt}}
\setcounter{secnumdepth}{-\maxdimen} % remove section numbering
\ifLuaTeX
  \usepackage{selnolig}  % disable illegal ligatures
\fi

\title{UE20CS312 - Data Analytics - Worksheet 2a - Simple Linear
Regression}
\author{GAURAV MAHAJAN}
\date{2022-09-15}

\begin{document}
\maketitle

\#\#Importing Libraries and Loading Dataset

\begin{Shaded}
\begin{Highlighting}[]
\FunctionTok{library}\NormalTok{(ggplot2)}
\end{Highlighting}
\end{Shaded}

\begin{verbatim}
## Warning: package 'ggplot2' was built under R version 4.2.1
\end{verbatim}

\begin{Shaded}
\begin{Highlighting}[]
\FunctionTok{library}\NormalTok{(dplyr)}
\end{Highlighting}
\end{Shaded}

\begin{verbatim}
## Warning: package 'dplyr' was built under R version 4.2.1
\end{verbatim}

\begin{verbatim}
## 
## Attaching package: 'dplyr'
\end{verbatim}

\begin{verbatim}
## The following objects are masked from 'package:stats':
## 
##     filter, lag
\end{verbatim}

\begin{verbatim}
## The following objects are masked from 'package:base':
## 
##     intersect, setdiff, setequal, union
\end{verbatim}

\begin{Shaded}
\begin{Highlighting}[]
\FunctionTok{library}\NormalTok{(tidyverse)}
\end{Highlighting}
\end{Shaded}

\begin{verbatim}
## Warning: package 'tidyverse' was built under R version 4.2.1
\end{verbatim}

\begin{verbatim}
## -- Attaching packages --------------------------------------- tidyverse 1.3.2 --
\end{verbatim}

\begin{verbatim}
## v tibble  3.1.8     v purrr   0.3.4
## v tidyr   1.2.0     v stringr 1.4.0
## v readr   2.1.2     v forcats 0.5.2
\end{verbatim}

\begin{verbatim}
## Warning: package 'tibble' was built under R version 4.2.1
\end{verbatim}

\begin{verbatim}
## Warning: package 'tidyr' was built under R version 4.2.1
\end{verbatim}

\begin{verbatim}
## Warning: package 'readr' was built under R version 4.2.1
\end{verbatim}

\begin{verbatim}
## Warning: package 'purrr' was built under R version 4.2.1
\end{verbatim}

\begin{verbatim}
## Warning: package 'stringr' was built under R version 4.2.1
\end{verbatim}

\begin{verbatim}
## Warning: package 'forcats' was built under R version 4.2.1
\end{verbatim}

\begin{verbatim}
## -- Conflicts ------------------------------------------ tidyverse_conflicts() --
## x dplyr::filter() masks stats::filter()
## x dplyr::lag()    masks stats::lag()
\end{verbatim}

\begin{Shaded}
\begin{Highlighting}[]
\NormalTok{df }\OtherTok{\textless{}{-}} \FunctionTok{read.csv}\NormalTok{(}\StringTok{"dragon\_neurons.csv"}\NormalTok{)}
\FunctionTok{head}\NormalTok{(df)}
\end{Highlighting}
\end{Shaded}

\begin{verbatim}
##   X axon_diameter conduction_velocity X.1
## 1 0            72            4.541130  NA
## 2 1            66            4.275300  NA
## 3 2            74            4.912093  NA
## 4 3             9            2.872806  NA
## 5 4             9            2.395194  NA
## 6 5            65            5.120160  NA
\end{verbatim}

\#\#\#Creating separate dataframes for diameter and conduction velocity

\begin{Shaded}
\begin{Highlighting}[]
\NormalTok{diameter }\OtherTok{\textless{}{-}}\NormalTok{ df}\SpecialCharTok{$}\StringTok{\textquotesingle{}axon\_diameter\textquotesingle{}}
\NormalTok{velocity }\OtherTok{\textless{}{-}}\NormalTok{ df}\SpecialCharTok{$}\StringTok{\textquotesingle{}conduction\_velocity\textquotesingle{}}
\end{Highlighting}
\end{Shaded}

\#\#\#Problem 1 (1 point) Find if a linear model is appropriate for
representing the relationship between the conduction velocity (response
variable) and axon diameter (explanatory variable) by finding the OLS
solution. Print out the slope and the coefficient. Plot the OLS best-fit
line of the model (Hint: use the ggplot library).

\begin{Shaded}
\begin{Highlighting}[]
\NormalTok{slope}\OtherTok{\textless{}{-}}\FunctionTok{cor}\NormalTok{(velocity,diameter)}\SpecialCharTok{*}\NormalTok{(}\FunctionTok{sd}\NormalTok{(velocity)}\SpecialCharTok{/}\FunctionTok{sd}\NormalTok{(diameter))}
\NormalTok{slope}
\end{Highlighting}
\end{Shaded}

\begin{verbatim}
## [1] 0.02475302
\end{verbatim}

\begin{Shaded}
\begin{Highlighting}[]
\NormalTok{intercept}\OtherTok{\textless{}{-}}\FunctionTok{mean}\NormalTok{(velocity)}\SpecialCharTok{{-}}\NormalTok{slope}\SpecialCharTok{*}\FunctionTok{mean}\NormalTok{(diameter)}
\NormalTok{intercept}
\end{Highlighting}
\end{Shaded}

\begin{verbatim}
## [1] 2.987611
\end{verbatim}

\begin{Shaded}
\begin{Highlighting}[]
\FunctionTok{plot}\NormalTok{(}\AttributeTok{x =}\NormalTok{diameter,}\AttributeTok{y=}\NormalTok{velocity,}\AttributeTok{main=}\StringTok{\textquotesingle{}Scatter Plot\textquotesingle{}}\NormalTok{,}\AttributeTok{xlab =} \StringTok{\textquotesingle{}Axon Diameter\textquotesingle{}}\NormalTok{, }\AttributeTok{ylab =} \StringTok{\textquotesingle{}Conduction Velocity\textquotesingle{}}\NormalTok{, }\AttributeTok{pch =} \DecValTok{16}\NormalTok{,}\AttributeTok{col=} \StringTok{"dark blue"}\NormalTok{)}
\FunctionTok{abline}\NormalTok{(}\FunctionTok{lm}\NormalTok{(velocity}\SpecialCharTok{\textasciitilde{}}\NormalTok{diameter),}\AttributeTok{col =}\StringTok{"dark gray"}\NormalTok{)}
\end{Highlighting}
\end{Shaded}

\includegraphics{PES1UG20CS150_WORKSHEET_2a_files/figure-latex/unnamed-chunk-3-1.pdf}
\#\#\#Problem 2 (3 points) Plot the residuals of the model. Do the
residuals look like white noise? If they do not, try to find a suitable
functional form (hint: try transforming either x or y using natural-log
or squares).

\begin{Shaded}
\begin{Highlighting}[]
\NormalTok{reg\_mod }\OtherTok{\textless{}{-}} \FunctionTok{lm}\NormalTok{(velocity }\SpecialCharTok{\textasciitilde{}}\NormalTok{ diameter, }\AttributeTok{data =}\NormalTok{ df)}
\NormalTok{res }\OtherTok{\textless{}{-}} \FunctionTok{resid}\NormalTok{(reg\_mod)}
\FunctionTok{plot}\NormalTok{(res, }\AttributeTok{type =} \StringTok{"b"}\NormalTok{) }\SpecialCharTok{+} \FunctionTok{abline}\NormalTok{(}\DecValTok{0}\NormalTok{, }\DecValTok{0}\NormalTok{)}
\end{Highlighting}
\end{Shaded}

\includegraphics{PES1UG20CS150_WORKSHEET_2a_files/figure-latex/unnamed-chunk-4-1.pdf}

\begin{verbatim}
## integer(0)
\end{verbatim}

No,the residuals do not seem to be of white noise kind.Hence going for
the natural log transformation

\begin{Shaded}
\begin{Highlighting}[]
\NormalTok{logmodel }\OtherTok{\textless{}{-}} \FunctionTok{lm}\NormalTok{(}\FunctionTok{log}\NormalTok{(conduction\_velocity) }\SpecialCharTok{\textasciitilde{}} \FunctionTok{log}\NormalTok{(axon\_diameter), }\AttributeTok{data =}\NormalTok{ df)}
\NormalTok{log\_res }\OtherTok{\textless{}{-}} \FunctionTok{resid}\NormalTok{(logmodel)}
\FunctionTok{plot}\NormalTok{(log\_res, }\AttributeTok{type =} \StringTok{"b"}\NormalTok{) }\SpecialCharTok{+} \FunctionTok{abline}\NormalTok{(}\DecValTok{0}\NormalTok{, }\DecValTok{0}\NormalTok{)}
\end{Highlighting}
\end{Shaded}

\includegraphics{PES1UG20CS150_WORKSHEET_2a_files/figure-latex/unnamed-chunk-5-1.pdf}

\begin{verbatim}
## integer(0)
\end{verbatim}

\#\#\#Problem 3 (3 points) Using Mahalanobis distance as a metric, are
there any potential outliers you notice? What are their Mahalanobis
distances? Use the model that you decided on in the previous problem
(Problem 2) as your regression model. Ensure that you plot the ellipse
with a radius equal to the square root of the Chi-square value with 2
degrees of freedom and 0.95 probability.

\begin{Shaded}
\begin{Highlighting}[]
\NormalTok{var }\OtherTok{\textless{}{-}}\NormalTok{ df[}\FunctionTok{c}\NormalTok{(}\StringTok{"axon\_diameter"}\NormalTok{, }\StringTok{"conduction\_velocity"}\NormalTok{)]}
\FunctionTok{na.omit}\NormalTok{(var)}
\end{Highlighting}
\end{Shaded}

\begin{verbatim}
##    axon_diameter conduction_velocity
## 1             72            4.541130
## 2             66            4.275300
## 3             74            4.912093
## 4              9            2.872806
## 5              9            2.395194
## 6             65            5.120160
## 7             79            4.457886
## 8             71            5.171822
## 9             58            4.254082
## 10            12            2.709620
## 11            52            4.325285
## 12            32            4.360330
## 13            60            4.712175
## 14            28            3.569837
## 15            46            4.033227
## 16            63            5.049463
## 17            48            4.307548
## 18            23            3.899511
## 19            55            4.421036
## 20            12            2.966179
## 21            65            5.142475
## 22            65            5.154147
## 23            55            4.924939
## 24            23            3.235679
## 25            56            4.776524
## 26            10            3.179216
## 27            90            4.726781
## 28            58            4.291057
## 29            58            4.379883
## 30            85            4.900027
## 31            38            3.938970
## 32            15            3.684369
## 33            88            5.072452
## 34            81            4.709887
## 35            65            4.630635
## 36            82            4.423754
## 37            89            4.825450
## 38            72            4.897530
## 39            73            5.255260
## 40            57            4.351882
## 41            50            4.780395
## 42            79            4.569002
## 43            61            4.260878
## 44            11            3.189608
## 45            27            3.865496
## 46             9            3.147883
## 47            76            5.298221
## 48            60            4.588713
## 49            10            3.205583
## 50            59            4.176738
## 51            63            4.908392
## 52            71            5.128692
## 53            38            3.674961
## 54            99            4.992772
## 55            24            3.956765
## 56            52            4.605855
## 57            89            4.824891
## 58            61            4.252118
## 59            94            4.877474
## 60            34            3.666817
## 61             8            2.979719
## 62            55            4.996601
## 63            77            4.883008
## 64            59            4.576930
## 65            56            4.138600
## 66            91            5.440959
## 67            38            3.685671
\end{verbatim}

\begin{Shaded}
\begin{Highlighting}[]
\NormalTok{var.center }\OtherTok{=} \FunctionTok{colMeans}\NormalTok{(var)}
\NormalTok{var.cov }\OtherTok{=} \FunctionTok{cov}\NormalTok{(var)}

\NormalTok{rad }\OtherTok{=} \FunctionTok{qchisq}\NormalTok{(}\AttributeTok{p =} \FloatTok{0.95}\NormalTok{, }\AttributeTok{df =} \DecValTok{2}\NormalTok{)}

\NormalTok{rad }\OtherTok{=} \FunctionTok{sqrt}\NormalTok{(rad)}

\NormalTok{ellipse }\OtherTok{\textless{}{-}}\NormalTok{ car}\SpecialCharTok{::}\FunctionTok{ellipse}\NormalTok{(}\AttributeTok{center =}\NormalTok{ var.center, }\AttributeTok{shape =}\NormalTok{ var.cov, }\AttributeTok{radius =}\NormalTok{ rad, }\AttributeTok{segments =} \DecValTok{150}\NormalTok{, }\AttributeTok{draw =} \ConstantTok{FALSE}\NormalTok{)}

\NormalTok{ellipse }\OtherTok{\textless{}{-}} \FunctionTok{as.data.frame}\NormalTok{(ellipse)}
\FunctionTok{colnames}\NormalTok{(ellipse) }\OtherTok{\textless{}{-}} \FunctionTok{colnames}\NormalTok{(var)}
\NormalTok{figure }\OtherTok{\textless{}{-}} \FunctionTok{ggplot}\NormalTok{(var , }\FunctionTok{aes}\NormalTok{(}\AttributeTok{x =}\NormalTok{ axon\_diameter , }\AttributeTok{y =}\NormalTok{ conduction\_velocity)) }\SpecialCharTok{+}
       \FunctionTok{geom\_point}\NormalTok{(}\AttributeTok{size =} \DecValTok{2}\NormalTok{) }\SpecialCharTok{+}
       \FunctionTok{geom\_polygon}\NormalTok{(}\AttributeTok{data =}\NormalTok{ ellipse , }\AttributeTok{fill =} \StringTok{"orange"}\NormalTok{ , }\AttributeTok{color =} \StringTok{"orange"}\NormalTok{ , }\AttributeTok{alpha =} \FloatTok{0.5}\NormalTok{)}\SpecialCharTok{+}
       \FunctionTok{geom\_point}\NormalTok{(}\FunctionTok{aes}\NormalTok{(var.center[}\DecValTok{1}\NormalTok{] , var.center[}\DecValTok{2}\NormalTok{]) , }\AttributeTok{size =} \DecValTok{5}\NormalTok{ , }\AttributeTok{color =} \StringTok{"blue"}\NormalTok{) }\SpecialCharTok{+}
       \FunctionTok{geom\_text}\NormalTok{( }\FunctionTok{aes}\NormalTok{(}\AttributeTok{label =} \FunctionTok{row.names}\NormalTok{(var)) , }\AttributeTok{hjust =} \DecValTok{1}\NormalTok{ , }\AttributeTok{vjust =} \SpecialCharTok{{-}}\FloatTok{1.5}\NormalTok{ ,}\AttributeTok{size =} \FloatTok{2.5}\NormalTok{ ) }\SpecialCharTok{+}
       \FunctionTok{ylab}\NormalTok{(}\StringTok{"Conduction Velocity"}\NormalTok{) }\SpecialCharTok{+} \FunctionTok{xlab}\NormalTok{(}\StringTok{"Axon Diameter"}\NormalTok{)}
\NormalTok{figure}
\end{Highlighting}
\end{Shaded}

\includegraphics{PES1UG20CS150_WORKSHEET_2a_files/figure-latex/unnamed-chunk-6-1.pdf}

\begin{Shaded}
\begin{Highlighting}[]
\NormalTok{distances }\OtherTok{\textless{}{-}} \FunctionTok{mahalanobis}\NormalTok{(}\AttributeTok{x =}\NormalTok{ var, }\AttributeTok{center =}\NormalTok{ var.center, }\AttributeTok{cov =}\NormalTok{ var.cov)}
\NormalTok{cutoff }\OtherTok{\textless{}{-}} \FunctionTok{qchisq}\NormalTok{(}\AttributeTok{p =} \FloatTok{0.95}\NormalTok{ , }\AttributeTok{df =} \DecValTok{2}\NormalTok{)}
\NormalTok{var[distances }\SpecialCharTok{\textgreater{}}\NormalTok{ cutoff ,]}
\end{Highlighting}
\end{Shaded}

\begin{verbatim}
##   axon_diameter conduction_velocity
## 5             9            2.395194
\end{verbatim}

\begin{Shaded}
\begin{Highlighting}[]
\FunctionTok{print}\NormalTok{(}\StringTok{"Distance of outlier from the center of the ellipse"}\NormalTok{)}
\end{Highlighting}
\end{Shaded}

\begin{verbatim}
## [1] "Distance of outlier from the center of the ellipse"
\end{verbatim}

\begin{Shaded}
\begin{Highlighting}[]
\NormalTok{distances[}\DecValTok{5}\NormalTok{]}
\end{Highlighting}
\end{Shaded}

\begin{verbatim}
## [1] 8.597339
\end{verbatim}

\#\#\#Problem 4 (1 point) What are the R-squared values of the initial
linear model and the functional form chosen in Problem 2? What do you
infer from this? (hint: use the summary function on the created linear
models)

\begin{Shaded}
\begin{Highlighting}[]
\FunctionTok{summary}\NormalTok{(reg\_mod)[}\DecValTok{8}\NormalTok{]}
\end{Highlighting}
\end{Shaded}

\begin{verbatim}
## $r.squared
## [1] 0.7656189
\end{verbatim}

\begin{Shaded}
\begin{Highlighting}[]
\FunctionTok{summary}\NormalTok{(logmodel)[}\DecValTok{8}\NormalTok{]}
\end{Highlighting}
\end{Shaded}

\begin{verbatim}
## $r.squared
## [1] 0.8370537
\end{verbatim}

From the higher r squared values of the functional model, we can infer
that the functional model is a better fit for our data.

\#\#\#Problem 5 (2 points) Using the same summary function as Problem 4,
determine if there is a statistically significant linear relationship at
a significance value of 0.05 of the overall model chosen in Problem 2.
What do you understand about the relationship between dragons' axon
diameters and conduction velocity? (Hint: understand the values
displayed in summary and search for the right data).

\begin{Shaded}
\begin{Highlighting}[]
\FunctionTok{summary}\NormalTok{(reg\_mod)}
\end{Highlighting}
\end{Shaded}

\begin{verbatim}
## 
## Call:
## lm(formula = velocity ~ diameter, data = df)
## 
## Residuals:
##      Min       1Q   Median       3Q      Max 
## -0.81519 -0.24935 -0.04665  0.32827  0.64757 
## 
## Coefficients:
##             Estimate Std. Error t value Pr(>|t|)    
## (Intercept) 2.987611   0.101069   29.56   <2e-16 ***
## diameter    0.024753   0.001699   14.57   <2e-16 ***
## ---
## Signif. codes:  0 '***' 0.001 '**' 0.01 '*' 0.05 '.' 0.1 ' ' 1
## 
## Residual standard error: 0.3509 on 65 degrees of freedom
## Multiple R-squared:  0.7656, Adjusted R-squared:  0.762 
## F-statistic: 212.3 on 1 and 65 DF,  p-value: < 2.2e-16
\end{verbatim}

\begin{Shaded}
\begin{Highlighting}[]
\FunctionTok{summary}\NormalTok{(logmodel)}
\end{Highlighting}
\end{Shaded}

\begin{verbatim}
## 
## Call:
## lm(formula = log(conduction_velocity) ~ log(axon_diameter), data = df)
## 
## Residuals:
##       Min        1Q    Median        3Q       Max 
## -0.193959 -0.059711 -0.003799  0.071776  0.115607 
## 
## Coefficients:
##                    Estimate Std. Error t value Pr(>|t|)    
## (Intercept)         0.54666    0.05017   10.90 2.62e-16 ***
## log(axon_diameter)  0.23701    0.01297   18.27  < 2e-16 ***
## ---
## Signif. codes:  0 '***' 0.001 '**' 0.01 '*' 0.05 '.' 0.1 ' ' 1
## 
## Residual standard error: 0.07465 on 65 degrees of freedom
## Multiple R-squared:  0.8371, Adjusted R-squared:  0.8345 
## F-statistic: 333.9 on 1 and 65 DF,  p-value: < 2.2e-16
\end{verbatim}

Because the p value is a lot lesser than 0.05, we can conclude there is
a significant linear relationship between x and y because the
correlation coefficient is significantly different from zero. Therefore,
there is a significant linear relationship between the dragons' axon
diameters and conduction velocity

\end{document}
