% Options for packages loaded elsewhere
\PassOptionsToPackage{unicode}{hyperref}
\PassOptionsToPackage{hyphens}{url}
%
\documentclass[
]{article}
\usepackage{amsmath,amssymb}
\usepackage{lmodern}
\usepackage{iftex}
\ifPDFTeX
  \usepackage[T1]{fontenc}
  \usepackage[utf8]{inputenc}
  \usepackage{textcomp} % provide euro and other symbols
\else % if luatex or xetex
  \usepackage{unicode-math}
  \defaultfontfeatures{Scale=MatchLowercase}
  \defaultfontfeatures[\rmfamily]{Ligatures=TeX,Scale=1}
\fi
% Use upquote if available, for straight quotes in verbatim environments
\IfFileExists{upquote.sty}{\usepackage{upquote}}{}
\IfFileExists{microtype.sty}{% use microtype if available
  \usepackage[]{microtype}
  \UseMicrotypeSet[protrusion]{basicmath} % disable protrusion for tt fonts
}{}
\makeatletter
\@ifundefined{KOMAClassName}{% if non-KOMA class
  \IfFileExists{parskip.sty}{%
    \usepackage{parskip}
  }{% else
    \setlength{\parindent}{0pt}
    \setlength{\parskip}{6pt plus 2pt minus 1pt}}
}{% if KOMA class
  \KOMAoptions{parskip=half}}
\makeatother
\usepackage{xcolor}
\IfFileExists{xurl.sty}{\usepackage{xurl}}{} % add URL line breaks if available
\IfFileExists{bookmark.sty}{\usepackage{bookmark}}{\usepackage{hyperref}}
\hypersetup{
  pdftitle={UE20CS312 - Data Analytics Worksheet 2b : Multiple Linear Regression},
  pdfauthor={GAURAV MAHAJAN},
  hidelinks,
  pdfcreator={LaTeX via pandoc}}
\urlstyle{same} % disable monospaced font for URLs
\usepackage[margin=1in]{geometry}
\usepackage{color}
\usepackage{fancyvrb}
\newcommand{\VerbBar}{|}
\newcommand{\VERB}{\Verb[commandchars=\\\{\}]}
\DefineVerbatimEnvironment{Highlighting}{Verbatim}{commandchars=\\\{\}}
% Add ',fontsize=\small' for more characters per line
\usepackage{framed}
\definecolor{shadecolor}{RGB}{248,248,248}
\newenvironment{Shaded}{\begin{snugshade}}{\end{snugshade}}
\newcommand{\AlertTok}[1]{\textcolor[rgb]{0.94,0.16,0.16}{#1}}
\newcommand{\AnnotationTok}[1]{\textcolor[rgb]{0.56,0.35,0.01}{\textbf{\textit{#1}}}}
\newcommand{\AttributeTok}[1]{\textcolor[rgb]{0.77,0.63,0.00}{#1}}
\newcommand{\BaseNTok}[1]{\textcolor[rgb]{0.00,0.00,0.81}{#1}}
\newcommand{\BuiltInTok}[1]{#1}
\newcommand{\CharTok}[1]{\textcolor[rgb]{0.31,0.60,0.02}{#1}}
\newcommand{\CommentTok}[1]{\textcolor[rgb]{0.56,0.35,0.01}{\textit{#1}}}
\newcommand{\CommentVarTok}[1]{\textcolor[rgb]{0.56,0.35,0.01}{\textbf{\textit{#1}}}}
\newcommand{\ConstantTok}[1]{\textcolor[rgb]{0.00,0.00,0.00}{#1}}
\newcommand{\ControlFlowTok}[1]{\textcolor[rgb]{0.13,0.29,0.53}{\textbf{#1}}}
\newcommand{\DataTypeTok}[1]{\textcolor[rgb]{0.13,0.29,0.53}{#1}}
\newcommand{\DecValTok}[1]{\textcolor[rgb]{0.00,0.00,0.81}{#1}}
\newcommand{\DocumentationTok}[1]{\textcolor[rgb]{0.56,0.35,0.01}{\textbf{\textit{#1}}}}
\newcommand{\ErrorTok}[1]{\textcolor[rgb]{0.64,0.00,0.00}{\textbf{#1}}}
\newcommand{\ExtensionTok}[1]{#1}
\newcommand{\FloatTok}[1]{\textcolor[rgb]{0.00,0.00,0.81}{#1}}
\newcommand{\FunctionTok}[1]{\textcolor[rgb]{0.00,0.00,0.00}{#1}}
\newcommand{\ImportTok}[1]{#1}
\newcommand{\InformationTok}[1]{\textcolor[rgb]{0.56,0.35,0.01}{\textbf{\textit{#1}}}}
\newcommand{\KeywordTok}[1]{\textcolor[rgb]{0.13,0.29,0.53}{\textbf{#1}}}
\newcommand{\NormalTok}[1]{#1}
\newcommand{\OperatorTok}[1]{\textcolor[rgb]{0.81,0.36,0.00}{\textbf{#1}}}
\newcommand{\OtherTok}[1]{\textcolor[rgb]{0.56,0.35,0.01}{#1}}
\newcommand{\PreprocessorTok}[1]{\textcolor[rgb]{0.56,0.35,0.01}{\textit{#1}}}
\newcommand{\RegionMarkerTok}[1]{#1}
\newcommand{\SpecialCharTok}[1]{\textcolor[rgb]{0.00,0.00,0.00}{#1}}
\newcommand{\SpecialStringTok}[1]{\textcolor[rgb]{0.31,0.60,0.02}{#1}}
\newcommand{\StringTok}[1]{\textcolor[rgb]{0.31,0.60,0.02}{#1}}
\newcommand{\VariableTok}[1]{\textcolor[rgb]{0.00,0.00,0.00}{#1}}
\newcommand{\VerbatimStringTok}[1]{\textcolor[rgb]{0.31,0.60,0.02}{#1}}
\newcommand{\WarningTok}[1]{\textcolor[rgb]{0.56,0.35,0.01}{\textbf{\textit{#1}}}}
\usepackage{graphicx}
\makeatletter
\def\maxwidth{\ifdim\Gin@nat@width>\linewidth\linewidth\else\Gin@nat@width\fi}
\def\maxheight{\ifdim\Gin@nat@height>\textheight\textheight\else\Gin@nat@height\fi}
\makeatother
% Scale images if necessary, so that they will not overflow the page
% margins by default, and it is still possible to overwrite the defaults
% using explicit options in \includegraphics[width, height, ...]{}
\setkeys{Gin}{width=\maxwidth,height=\maxheight,keepaspectratio}
% Set default figure placement to htbp
\makeatletter
\def\fps@figure{htbp}
\makeatother
\setlength{\emergencystretch}{3em} % prevent overfull lines
\providecommand{\tightlist}{%
  \setlength{\itemsep}{0pt}\setlength{\parskip}{0pt}}
\setcounter{secnumdepth}{-\maxdimen} % remove section numbering
\ifLuaTeX
  \usepackage{selnolig}  % disable illegal ligatures
\fi

\title{UE20CS312 - Data Analytics Worksheet 2b : Multiple Linear
Regression}
\author{GAURAV MAHAJAN}
\date{2022-09-15}

\begin{document}
\maketitle

\#\#\#Importing libraries and uploading the dataset

\begin{Shaded}
\begin{Highlighting}[]
\FunctionTok{library}\NormalTok{(tidyverse)}
\end{Highlighting}
\end{Shaded}

\begin{verbatim}
## Warning: package 'tidyverse' was built under R version 4.2.1
\end{verbatim}

\begin{verbatim}
## -- Attaching packages --------------------------------------- tidyverse 1.3.2 --
## v ggplot2 3.3.6     v purrr   0.3.4
## v tibble  3.1.8     v dplyr   1.0.9
## v tidyr   1.2.0     v stringr 1.4.0
## v readr   2.1.2     v forcats 0.5.2
\end{verbatim}

\begin{verbatim}
## Warning: package 'ggplot2' was built under R version 4.2.1
\end{verbatim}

\begin{verbatim}
## Warning: package 'tibble' was built under R version 4.2.1
\end{verbatim}

\begin{verbatim}
## Warning: package 'tidyr' was built under R version 4.2.1
\end{verbatim}

\begin{verbatim}
## Warning: package 'readr' was built under R version 4.2.1
\end{verbatim}

\begin{verbatim}
## Warning: package 'purrr' was built under R version 4.2.1
\end{verbatim}

\begin{verbatim}
## Warning: package 'dplyr' was built under R version 4.2.1
\end{verbatim}

\begin{verbatim}
## Warning: package 'stringr' was built under R version 4.2.1
\end{verbatim}

\begin{verbatim}
## Warning: package 'forcats' was built under R version 4.2.1
\end{verbatim}

\begin{verbatim}
## -- Conflicts ------------------------------------------ tidyverse_conflicts() --
## x dplyr::filter() masks stats::filter()
## x dplyr::lag()    masks stats::lag()
\end{verbatim}

\begin{Shaded}
\begin{Highlighting}[]
\FunctionTok{library}\NormalTok{(corrplot)}
\end{Highlighting}
\end{Shaded}

\begin{verbatim}
## Warning: package 'corrplot' was built under R version 4.2.1
\end{verbatim}

\begin{verbatim}
## corrplot 0.92 loaded
\end{verbatim}

\begin{Shaded}
\begin{Highlighting}[]
\FunctionTok{library}\NormalTok{(olsrr)}
\end{Highlighting}
\end{Shaded}

\begin{verbatim}
## Warning: package 'olsrr' was built under R version 4.2.1
\end{verbatim}

\begin{verbatim}
## 
## Attaching package: 'olsrr'
## 
## The following object is masked from 'package:datasets':
## 
##     rivers
\end{verbatim}

\begin{Shaded}
\begin{Highlighting}[]
\NormalTok{df }\OtherTok{\textless{}{-}} \FunctionTok{read\_csv}\NormalTok{(}\StringTok{\textquotesingle{}spotify.csv\textquotesingle{}}\NormalTok{)}
\end{Highlighting}
\end{Shaded}

\begin{verbatim}
## Rows: 195 Columns: 13
## -- Column specification --------------------------------------------------------
## Delimiter: ","
## dbl (13): danceability, energy, key, loudness, mode, speechiness, acousticne...
## 
## i Use `spec()` to retrieve the full column specification for this data.
## i Specify the column types or set `show_col_types = FALSE` to quiet this message.
\end{verbatim}

\begin{Shaded}
\begin{Highlighting}[]
\FunctionTok{head}\NormalTok{(df)}
\end{Highlighting}
\end{Shaded}

\begin{verbatim}
## # A tibble: 6 x 13
##   danceabil~1 energy   key loudn~2  mode speec~3 acous~4 instr~5 liven~6 valence
##         <dbl>  <dbl> <dbl>   <dbl> <dbl>   <dbl>   <dbl>   <dbl>   <dbl>   <dbl>
## 1       0.803 0.624      7   -6.76     0  0.0477  0.451  7.34e-4  0.1     0.628 
## 2       0.762 0.703     10   -7.95     0  0.306   0.206  0        0.0912  0.519 
## 3       0.261 0.0149     1  -27.5      1  0.0419  0.992  8.97e-1  0.102   0.0382
## 4       0.722 0.736      3   -6.99     0  0.0585  0.431  1.18e-6  0.123   0.582 
## 5       0.787 0.572      1   -7.52     1  0.222   0.145  0        0.0753  0.647 
## 6       0.778 0.632      8   -6.42     1  0.125   0.0404 0        0.0912  0.827 
## # ... with 3 more variables: tempo <dbl>, duration_ms <dbl>,
## #   time_signature <dbl>, and abbreviated variable names 1: danceability,
## #   2: loudness, 3: speechiness, 4: acousticness, 5: instrumentalness,
## #   6: liveness
## # i Use `colnames()` to see all variable names
\end{verbatim}

\#\#\#Problem-1 (0.5 Points) Check for missing values in the dataset and
normalize the dataset.

\begin{Shaded}
\begin{Highlighting}[]
\CommentTok{\#checking for missing values}
\FunctionTok{sum}\NormalTok{(}\FunctionTok{is.na}\NormalTok{(df))}
\end{Highlighting}
\end{Shaded}

\begin{verbatim}
## [1] 0
\end{verbatim}

\begin{Shaded}
\begin{Highlighting}[]
\CommentTok{\#Normalisation}
\NormalTok{min\_max\_norm }\OtherTok{\textless{}{-}} \ControlFlowTok{function}\NormalTok{(x) \{}
\NormalTok{  (x }\SpecialCharTok{{-}} \FunctionTok{min}\NormalTok{(x)) }\SpecialCharTok{/}\NormalTok{ (}\FunctionTok{max}\NormalTok{(x) }\SpecialCharTok{{-}} \FunctionTok{min}\NormalTok{(x))}
\NormalTok{\}}
\NormalTok{df\_norm }\OtherTok{\textless{}{-}} \FunctionTok{as.data.frame}\NormalTok{(}\FunctionTok{lapply}\NormalTok{(df, min\_max\_norm))}
\end{Highlighting}
\end{Shaded}

This implies there is no missing data in the dataset

\begin{Shaded}
\begin{Highlighting}[]
\CommentTok{\#for scaling : }
\CommentTok{\#for z score scaling to be done centering is done}
\NormalTok{df}\OtherTok{\textless{}{-}}\FunctionTok{as.data.frame}\NormalTok{(}\FunctionTok{scale}\NormalTok{(df))}
\end{Highlighting}
\end{Shaded}

\#\#\#Problem-2 (2 Points) Fit a linear model to predict the energy
rating using all other attributes.Get the summary of the model and
explain the results in detail.{[}Hint : Use the lm() function. Click
here To get the documentation of the same.{]}

\begin{Shaded}
\begin{Highlighting}[]
\CommentTok{\#For all the attributes fitting a linear model}
\NormalTok{full\_model}\OtherTok{\textless{}{-}}\FunctionTok{lm}\NormalTok{(energy}\SpecialCharTok{\textasciitilde{}}\NormalTok{.,}\AttributeTok{data =}\NormalTok{ df)}
\FunctionTok{summary}\NormalTok{(full\_model)}
\end{Highlighting}
\end{Shaded}

\begin{verbatim}
## 
## Call:
## lm(formula = energy ~ ., data = df)
## 
## Residuals:
##      Min       1Q   Median       3Q      Max 
## -1.00232 -0.22889 -0.00973  0.27796  1.24597 
## 
## Coefficients:
##                    Estimate Std. Error t value Pr(>|t|)    
## (Intercept)       9.156e-17  2.920e-02   0.000  1.00000    
## danceability     -2.751e-01  5.555e-02  -4.952 1.67e-06 ***
## key               4.970e-02  3.009e-02   1.652  0.10030    
## loudness          7.015e-01  4.561e-02  15.381  < 2e-16 ***
## mode             -4.794e-02  3.034e-02  -1.580  0.11582    
## speechiness       2.359e-02  3.519e-02   0.670  0.50343    
## acousticness     -3.435e-01  4.136e-02  -8.306 2.21e-14 ***
## instrumentalness  1.493e-01  5.577e-02   2.677  0.00811 ** 
## liveness          2.004e-02  3.100e-02   0.646  0.51880    
## valence           2.046e-01  3.884e-02   5.269 3.85e-07 ***
## tempo            -2.395e-02  3.295e-02  -0.727  0.46817    
## duration_ms      -1.865e-02  3.303e-02  -0.565  0.57298    
## time_signature    2.409e-02  3.220e-02   0.748  0.45535    
## ---
## Signif. codes:  0 '***' 0.001 '**' 0.01 '*' 0.05 '.' 0.1 ' ' 1
## 
## Residual standard error: 0.4077 on 182 degrees of freedom
## Multiple R-squared:  0.844,  Adjusted R-squared:  0.8338 
## F-statistic: 82.08 on 12 and 182 DF,  p-value: < 2.2e-16
\end{verbatim}

The asterisks tell us the significance .If alpha is 0.05 :: we select :
danceability , loudness , acousticness , instrumentalness , valence
according to this model.Betas are not all zero seeing he F statistic

\#\#\#Problem-3 (2 points) With the help of a correlogram and scatter
plots, choose the features you think are important and model an MLR.
Justify your choice and explain the new findings.

\begin{Shaded}
\begin{Highlighting}[]
\NormalTok{df\_cor }\OtherTok{\textless{}{-}} \FunctionTok{cor}\NormalTok{(df\_norm)}
\FunctionTok{corrplot}\NormalTok{(df\_cor, }\AttributeTok{order =} \StringTok{"hclust"}\NormalTok{, }\AttributeTok{method =} \StringTok{"number"}\NormalTok{)}
\end{Highlighting}
\end{Shaded}

\includegraphics{PES1UG20CS150_WORKSHEET_2b_files/figure-latex/unnamed-chunk-5-1.pdf}

\begin{Shaded}
\begin{Highlighting}[]
\FunctionTok{plot}\NormalTok{(df\_norm)}
\end{Highlighting}
\end{Shaded}

\includegraphics{PES1UG20CS150_WORKSHEET_2b_files/figure-latex/unnamed-chunk-5-2.pdf}

\begin{Shaded}
\begin{Highlighting}[]
\CommentTok{\#energy has significant positive correlation with loudness, valence and tempo}
\CommentTok{\#energy has significant negative correlation with acousticness, instrumentalness}
\CommentTok{\#these variables are significant in the model}

\NormalTok{model\_cor }\OtherTok{\textless{}{-}} \FunctionTok{lm}\NormalTok{(energy}\SpecialCharTok{\textasciitilde{}}\NormalTok{loudness}\SpecialCharTok{+}\NormalTok{acousticness}\SpecialCharTok{+}\NormalTok{valence}\SpecialCharTok{+}\NormalTok{tempo}\SpecialCharTok{+}\NormalTok{instrumentalness, }\AttributeTok{data=}\NormalTok{df\_norm)}
\FunctionTok{summary}\NormalTok{(model\_cor)}
\end{Highlighting}
\end{Shaded}

\begin{verbatim}
## 
## Call:
## lm(formula = energy ~ loudness + acousticness + valence + tempo + 
##     instrumentalness, data = df_norm)
## 
## Residuals:
##      Min       1Q   Median       3Q      Max 
## -0.29492 -0.07908  0.00527  0.08305  0.33311 
## 
## Coefficients:
##                   Estimate Std. Error t value Pr(>|t|)    
## (Intercept)      -0.282319   0.080630  -3.501 0.000577 ***
## loudness          1.111635   0.077476  14.348  < 2e-16 ***
## acousticness     -0.294210   0.035350  -8.323 1.69e-14 ***
## valence           0.123858   0.036325   3.410 0.000795 ***
## tempo            -0.003382   0.037722  -0.090 0.928645    
## instrumentalness  0.230480   0.032243   7.148 1.86e-11 ***
## ---
## Signif. codes:  0 '***' 0.001 '**' 0.01 '*' 0.05 '.' 0.1 ' ' 1
## 
## Residual standard error: 0.1154 on 189 degrees of freedom
## Multiple R-squared:  0.8108, Adjusted R-squared:  0.8058 
## F-statistic:   162 on 5 and 189 DF,  p-value: < 2.2e-16
\end{verbatim}

\#\#\#Problem-4 (1 Point) Conduct a partial F-test to determine if the
attributes not chosen by you in Problem-3 are significant to predict the
energy.What are the null and alternate hypotheses? {[} Hint : Use the
anova function between models created in Problem-2 and Problem-3 {]}

\begin{Shaded}
\begin{Highlighting}[]
\FunctionTok{anova}\NormalTok{(model\_cor,full\_model)}
\end{Highlighting}
\end{Shaded}

\begin{verbatim}
## Analysis of Variance Table
## 
## Model 1: energy ~ loudness + acousticness + valence + tempo + instrumentalness
## Model 2: energy ~ danceability + key + loudness + mode + speechiness + 
##     acousticness + instrumentalness + liveness + valence + tempo + 
##     duration_ms + time_signature
##   Res.Df     RSS Df Sum of Sq F Pr(>F)
## 1    189  2.5151                      
## 2    182 30.2566  7   -27.741
\end{verbatim}

\#\#\#Problem-5 (1.5 Points) AIC - Akaike Information Criterion is used
to compare different models and determine the best fit for the data. The
best-fit model according to AIC is the one that explains greatest amount
of variation using the fewest number of attributes. Check this resource
to learn more about AIC. Build a model based on AIC using Stepwise AIC
regression.Elucidate your observations from the new model. ( Hint : Use
an appropriate function in olsrr package.)

\begin{Shaded}
\begin{Highlighting}[]
\NormalTok{stepwise\_model}\OtherTok{\textless{}{-}}\FunctionTok{lm}\NormalTok{(energy }\SpecialCharTok{\textasciitilde{}}\NormalTok{ loudness }\SpecialCharTok{+}\NormalTok{ acousticness }\SpecialCharTok{+}\NormalTok{ danceability }\SpecialCharTok{+}\NormalTok{ valence }\SpecialCharTok{+}\NormalTok{ instrumentalness }\SpecialCharTok{+}\NormalTok{ mode }\SpecialCharTok{+}\NormalTok{ key , }\AttributeTok{data=}\NormalTok{df)}
\FunctionTok{summary}\NormalTok{(stepwise\_model)}
\end{Highlighting}
\end{Shaded}

\begin{verbatim}
## 
## Call:
## lm(formula = energy ~ loudness + acousticness + danceability + 
##     valence + instrumentalness + mode + key, data = df)
## 
## Residuals:
##      Min       1Q   Median       3Q      Max 
## -1.05662 -0.24874 -0.01126  0.27930  1.25974 
## 
## Coefficients:
##                    Estimate Std. Error t value Pr(>|t|)    
## (Intercept)       9.999e-17  2.900e-02   0.000  1.00000    
## loudness          7.075e-01  4.462e-02  15.856  < 2e-16 ***
## acousticness     -3.420e-01  4.005e-02  -8.539 4.63e-15 ***
## danceability     -2.681e-01  5.308e-02  -5.051 1.04e-06 ***
## valence           2.003e-01  3.825e-02   5.238 4.35e-07 ***
## instrumentalness  1.418e-01  5.351e-02   2.650  0.00873 ** 
## mode             -4.863e-02  2.985e-02  -1.629  0.10491    
## key               4.488e-02  2.950e-02   1.521  0.12988    
## ---
## Signif. codes:  0 '***' 0.001 '**' 0.01 '*' 0.05 '.' 0.1 ' ' 1
## 
## Residual standard error: 0.4049 on 187 degrees of freedom
## Multiple R-squared:  0.842,  Adjusted R-squared:  0.8361 
## F-statistic: 142.3 on 7 and 187 DF,  p-value: < 2.2e-16
\end{verbatim}

\#\#\#Problem-6 (1 Point) Plot the residuals of the models built till
now and comment on it satisfying the assumptions of MLR.

\begin{Shaded}
\begin{Highlighting}[]
\FunctionTok{plot}\NormalTok{(full\_model}\SpecialCharTok{$}\NormalTok{residuals , }\AttributeTok{pch =} \DecValTok{16}\NormalTok{, }\AttributeTok{col=}\StringTok{"red"}\NormalTok{)}
\FunctionTok{abline}\NormalTok{(}\AttributeTok{h=}\DecValTok{0}\NormalTok{,}\AttributeTok{lty=}\DecValTok{2}\NormalTok{)}
\end{Highlighting}
\end{Shaded}

\includegraphics{PES1UG20CS150_WORKSHEET_2b_files/figure-latex/unnamed-chunk-8-1.pdf}

\begin{Shaded}
\begin{Highlighting}[]
\FunctionTok{ols\_plot\_resid\_hist}\NormalTok{(full\_model)}
\end{Highlighting}
\end{Shaded}

\includegraphics{PES1UG20CS150_WORKSHEET_2b_files/figure-latex/unnamed-chunk-8-2.pdf}

\begin{Shaded}
\begin{Highlighting}[]
\FunctionTok{plot}\NormalTok{(model\_cor}\SpecialCharTok{$}\NormalTok{residuals , }\AttributeTok{pch =} \DecValTok{16}\NormalTok{, }\AttributeTok{col=}\StringTok{"red"}\NormalTok{)}
\FunctionTok{abline}\NormalTok{(}\AttributeTok{h=}\DecValTok{0}\NormalTok{,}\AttributeTok{lty=}\DecValTok{2}\NormalTok{)}
\end{Highlighting}
\end{Shaded}

\includegraphics{PES1UG20CS150_WORKSHEET_2b_files/figure-latex/unnamed-chunk-9-1.pdf}

\begin{Shaded}
\begin{Highlighting}[]
\FunctionTok{ols\_plot\_resid\_hist}\NormalTok{(model\_cor)}
\end{Highlighting}
\end{Shaded}

\includegraphics{PES1UG20CS150_WORKSHEET_2b_files/figure-latex/unnamed-chunk-9-2.pdf}

Problem-7 (2 Points) For the model built in Problem-2 , determine the
presence of multicollinearity using VIF. Determine if there are outliers
in the data using Cook's Distance. If you find any , remove the outliers
and fit the model for Problem-2 and see if the fit improves. {[} Hint :
All the relevant functions can be found in olsrr package. An observation
can be termed as an outlier if it has a Cook's distance of more than 4/n
where n is the number of records.{]}

\begin{Shaded}
\begin{Highlighting}[]
\FunctionTok{ols\_vif\_tol}\NormalTok{(full\_model)}
\end{Highlighting}
\end{Shaded}

\begin{verbatim}
##           Variables Tolerance      VIF
## 1      danceability 0.2776703 3.601393
## 2               key 0.9467671 1.056226
## 3          loudness 0.4119898 2.427245
## 4              mode 0.9308390 1.074300
## 5       speechiness 0.6921660 1.444740
## 6      acousticness 0.5009458 1.996224
## 7  instrumentalness 0.2755568 3.629016
## 8          liveness 0.8914397 1.121781
## 9           valence 0.5680642 1.760364
## 10            tempo 0.7892957 1.266952
## 11      duration_ms 0.7855373 1.273014
## 12   time_signature 0.8262918 1.210226
\end{verbatim}

\begin{Shaded}
\begin{Highlighting}[]
\NormalTok{cooks }\OtherTok{\textless{}{-}} \FunctionTok{ols\_plot\_cooksd\_chart}\NormalTok{(full\_model)}
\end{Highlighting}
\end{Shaded}

\includegraphics{PES1UG20CS150_WORKSHEET_2b_files/figure-latex/unnamed-chunk-10-1.pdf}

\begin{Shaded}
\begin{Highlighting}[]
\NormalTok{new\_df}\OtherTok{\textless{}{-}}\NormalTok{df[}\SpecialCharTok{{-}}\FunctionTok{c}\NormalTok{(}\DecValTok{30}\NormalTok{,}\DecValTok{35}\NormalTok{,}\DecValTok{49}\NormalTok{,}\DecValTok{84}\NormalTok{,}\DecValTok{114}\NormalTok{,}\DecValTok{120}\NormalTok{,}\DecValTok{127}\NormalTok{,}\DecValTok{144}\NormalTok{,}\DecValTok{153}\NormalTok{,}\DecValTok{159}\NormalTok{,}\DecValTok{171}\NormalTok{,}\DecValTok{172}\NormalTok{,}\DecValTok{187}\NormalTok{,}\DecValTok{191}\NormalTok{),]}
\NormalTok{new\_full\_model}\OtherTok{\textless{}{-}}\FunctionTok{lm}\NormalTok{(energy}\SpecialCharTok{\textasciitilde{}}\NormalTok{.,}\AttributeTok{data=}\NormalTok{new\_df)}
\FunctionTok{summary}\NormalTok{(new\_full\_model)}
\end{Highlighting}
\end{Shaded}

\begin{verbatim}
## 
## Call:
## lm(formula = energy ~ ., data = new_df)
## 
## Residuals:
##      Min       1Q   Median       3Q      Max 
## -0.76364 -0.20836  0.01581  0.23506  0.95145 
## 
## Coefficients:
##                   Estimate Std. Error t value Pr(>|t|)    
## (Intercept)      -0.001128   0.025283  -0.045 0.964458    
## danceability     -0.258483   0.052291  -4.943 1.85e-06 ***
## key               0.088181   0.026094   3.379 0.000903 ***
## loudness          0.838411   0.045399  18.468  < 2e-16 ***
## mode             -0.012666   0.026559  -0.477 0.634036    
## speechiness      -0.004528   0.032087  -0.141 0.887947    
## acousticness     -0.280188   0.037293  -7.513 3.26e-12 ***
## instrumentalness  0.199483   0.051442   3.878 0.000151 ***
## liveness          0.028416   0.027232   1.043 0.298230    
## valence           0.187216   0.033329   5.617 7.90e-08 ***
## tempo            -0.018193   0.029627  -0.614 0.540008    
## duration_ms      -0.059788   0.028685  -2.084 0.038647 *  
## time_signature    0.036680   0.028430   1.290 0.198761    
## ---
## Signif. codes:  0 '***' 0.001 '**' 0.01 '*' 0.05 '.' 0.1 ' ' 1
## 
## Residual standard error: 0.337 on 168 degrees of freedom
## Multiple R-squared:  0.8778, Adjusted R-squared:  0.8691 
## F-statistic: 100.6 on 12 and 168 DF,  p-value: < 2.2e-16
\end{verbatim}

\end{document}
